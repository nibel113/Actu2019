\chapter{GRF-2}\label{grf-2}

\section{Bases}


\subsection*{Produit dérivé}\label{produit-derive}
\addcontentsline{toc}{subsection}{Produit dérivé}

Contrat entre 2 partis qui fixe les flux monétaires futurs fondés sur
ceux de l'actif sous-jacent(SJ).

\subsection*{Étapes d'une transaction}\label{etapes-dune-transaction}
\addcontentsline{toc}{subsection}{Étapes d'une transaction}

\begin{enumerate}
\def\labelenumi{\arabic{enumi}.}

\item
  Acheteur et vendeur se trouve. Facilité par la bourse.
\item
  Obligations pour les deux partis définis (prix, produits, conditions).
  Si transaction à la bourse: intermédiaire, et donc, dépôts de garantie
  possibles.
\item
  Transaction
\item
  Mise à jour du registre de propriété.
\end{enumerate}

\subsection*{Mesure d'évaluation taille(activité) de la bourse(marché)}\label{mesure-devaluation-tailleactivite-de-la-boursemarche}
\addcontentsline{toc}{subsection}{Mesure d'évaluation taille(activité) de la bourse(marché)}

\begin{itemize}

\item
  Volume de transaction: nombre de titres transigés parpériodes
\item
  Valeur marchande: Valeur d'une action/cie/indice boursier
\item
  Positions ouvertes: quantité de contrats qui ne sont pas arrivés à
  échéance
\end{itemize}

\subsection*{Rôle des marchés financiers}\label{role-des-marches-financiers}
\addcontentsline{toc}{subsection}{Rôle des marchés financiers}

\begin{itemize}

\item
  Partage du risque: compagnie partage le risque et les profits avec les
  actionnaires
\item
  Diversification du risque: risque diversifiable \(\rightarrow\)
  théoriquement possible de diluer le risque pour qu'il devienne nul.
  Risque non-diversifiable \(\rightarrow\) possible de transférer le
  risque via des produits dérivés.
\end{itemize}

\subsection*{Utilité des produits dérivés}\label{utilite-des-produits-derives}
\addcontentsline{toc}{subsection}{Utilité des produits dérivés}

\begin{itemize}

\item
  Gestion des risques
\item
  Spéculation
\item
  Réduction des frais de transaction
\item
  arbitrage réglementaire
\end{itemize}

\subsection*{3 types d'acteurs}\label{types-dacteurs}
\addcontentsline{toc}{subsection}{3 types d'acteurs}

\begin{itemize}

\item
  Uilisateurs(acheteur/vendeurs)
\item
  Teneur de marché(intermédiaire)
\item
  Observateur(analyste/autorité)
\end{itemize}

\subsection*{Définitions}\label{definitions-1}
\addcontentsline{toc}{subsection}{Définitions}

\begin{itemize}

\item
  Ordre au cours du marché: quantité de l'actif visé à acheter(vendre)
  au prix du marché, au moment où l'ordre est passée.
\item
  Ordre à cours limité: quantité d'actions à acheter/vendre dans une
  tranche spécifique de prix.
\item
  Ordre de vente «stop»: prix en dessous duquel on vend automatiquement.
\item
  Position longue: qui profitera de l'augmentation de la valeur du SJ.
\item
  Position courte: qui profitera de la diminution de la valeur du SJ.
\item
  Vente à découvert: vente d'un actif qu'on ne possède pas. L'actif est
  livré à une date ultérieur, mais paiement à t=0 au prix de l'actif à
  t=0.

  \begin{itemize}
  
  \item
    Utilité:

    \begin{itemize}
    
    \item
      Spéculation
    \item
      Financement
    \item
      Couverture contre la baisse de valeur
    \end{itemize}
  \item
    Risque:

    \begin{itemize}
    
    \item
      de défaut
    \item
      de rareté
    \end{itemize}
  \end{itemize}
\end{itemize}

\section{CAPM(Capital asset pricing management)}\label{capmcapital-asset-pricing-management}

\subsection*{3 postulats:}\label{postulats}

\begin{enumerate}
\def\labelenumi{\arabic{enumi}.}

\item
  Transactions efficaces et sans friction: pas de frais de transaction,
  emprunt au taux sans risque.
\item
  Rationnalité des investisseurs: maximise leur ratio de Sharpe \[
  \rightarrow\; \frac{E[R_p- r_f]}{\sigma_p}
  \]
\item
  Attentes et espérances homogènes
\end{enumerate}

L'équation du CAPM pour un actif \(\mathrm{i}\):

\[
R_i =r_f+a_i+\beta_i(R_{mkt}-r_f)+\epsilon\\  
\] Cela implique:\\
\[
\frac{dR_i}{d R_{mkt}}=\beta_i=\frac{Cov(R_i,R_{mkt})}{Var(R_{mkt})}\\    
\] Pour un portefeuille p:\\
\[
\frac{dR_p}{d R_{mkt}}=\beta_p=\frac{Cov(\sum x_i R_i,R_{mkt})}{Var(R_{mkt})}=\sum x_i \frac{Cov(R_i,R_{mkt})}{Var(R_{mkt})}=\sum x_i \beta_i
\]

\subsection*{Incohérences du modèle}\label{incoherences-du-modele}

\begin{itemize}

\item
  Investisseurs non rationnels et pas informés sur leur portefeuille
\item
  Certains ne veulent pas nécessairement maximiser leur ratio de Sharpe,
  ont d'autres objectifs
\item
  Il y a des investisseurs qui ne diversifient pas leur portefeuille de
  manière optimale
\item
  Il y en a qui sont ultra-actif, malgré le fait que le CAPM suppose une
  gestion passive
\end{itemize}

Comportements avec effet plus systémique:

\begin{itemize}

\item
  Peur du regret: garder un titre qui est en train de baisser ou vendre
  un titre avant qu'il remonte
\item
  Les investisseurs sont influençables; ils achèteront les titres
  médiatisés,etc.
\item
  Effet de trouppeau: on fait comme ceux qu'on connait
\end{itemize}

\subsection*{Modèle multifactoriel et l'APT(arbitrage pricing theory)}\label{modele-multifactoriel-et-laptarbitrage-pricing-theory}

Trois types d'actifs avec des alphas strictement positifs qui
contredisent le CAPM:

\begin{itemize}

\item
  Petites capitalisations: on observe des rendements supérieurs à ce que
  le CAPM prédit
\item
  Book to market ratio: titres ``value'' avec une valeur au livre
  supérieur à la valeur marchande verront la valeur marchande rejoindre
  la valeur au livre avec le temps
\item
  Momemtum: les compagnies qui ont connues un bon rendement dernièrement
  auront tendance à avoir un rendement supérieur à la moyenne
\end{itemize}

\subsubsection*{APT}\label{apt}

\[
E[R_s]-r_f= \sum_{i=1}^N \beta_s^{Fi}(E[R_{Fi}]-r_f)
\] Les ``F'' sont des facteurs. Il est possible de créer des modèles
avec n'importe quels facteurs comme des indices boursiers.

\section{Foward et options}\label{fowardetopt}

\subsection{Contrat Foward}\label{contrat-foward}

Achat d'un actif prédeterminé à une valeur initiale \(S_0\), à une date
de livraison \(T\) et à un prix \(F_{0,T}\). Le coût initial est nul.
\(F_{0,T}\) est le prix anticipé de l'acftif sous-jacent rendu à la date
\(T\). \(S_0(1+r_f)^T=F_{0,T}\)

\begin{itemize}

\item
  Valeur à l'échéance:

  \begin{itemize}
  
  \item
    Pour l'acheteur(position longue): \(F_{0,T} - S_T\)
  \item
    Pour le vendeur(position courte): \(S_t - F_{0,T}\)
  \end{itemize}
\end{itemize}


\includegraphics{_bookdown_files/03-GRF-2_files/figure-latex/unnamed-chunk-1-1.pdf} 


\subsection{Foward prépayé}\label{foward-prepaye}

Dans certain cas, l'acheteur voudra payé à \(t=0\). Le coût initial sera
\(F_{0,T}^P\). On achète immédiatemment sans avoir l'actif à la date de
transaction. La position de l'acheteur est \emph{capitalisée}. Dans un
achat ferme, la position de l'acheteur est pleinement capitalisée. Le
contrat foward, lui, implique une position non capitalisée.\\
\\


\begin{tabularx}{0.5\textwidth}{Xrr}
\hline
Temps & Acheteur & Vendeur \\
\hline
$t=0$ & $-F_{0,T}^P$ & $F_{0,T}^P$ \\
$t=T$ & $S_T$ & $-S_T$ \\
\hline
\end{tabularx}

\includegraphics[scale=1]{_bookdown_files/03-GRF-2_files/figure-latex/unnamed-chunk-3-1.pdf} 

Pour recréer les cashflows d'un contrat foward avec un achat ferme, on
finance l'achat ferme avec un emprunt au taux sans risque.\\
\\


\begin{tabular}{lrrr}
\hline
Temps & Achat ferme & + Emprunt & = Foward \\
\hline
$t=0$ & $-S_0$ & $S_0$ & $\varnothing$ \\
$t=T$ & $S_T$ & $F_{0,T}$ & $S_T - F_{0,T}$ \\
\hline
\end{tabular}\\
\\



On peut aussi recréer les cashflows d'un achat ferme avec un foward et
en investissant la valeur actualisée de \(F_{0,T}\).\\
\(F_{0,T}(1+r_f)^{-T}=S_0(1+r_f)^{T}(1+r_f)^{-T}=S_0\).\\
\\


\begin{tabular}{lrrr}
\hline
Temps & Dépot & + Foward & = Achat ferme \\
\hline
$t=0$ & $-S_0$ & $\varnothing$ & $-S_0$ \\
$t=T$ & $F_{0,T}$ & $S_T - F_{0,T}$ & $S_T$ \\
\hline
\end{tabular}

\subsection{Option d'achat(call)}\label{option-dachatcall}

Contrat qui permet au détenteur(position longue) d'acheter un actif
sous-jacent à un prix prédéterminé, strike price \(=K\), à une date
d'échéance ou d'içi cette date, s'il le désire.\\
\\


3 types de levées:

\begin{enumerate}
\def\labelenumi{\arabic{enumi}.}

\item
  Européenne (à la date T)
\item
  Américaine (d'içi la date T)
\item
  Bermudienne (à certains moments d'içi T)
\end{enumerate}

\begin{tabular}{lrr}
\hline
\multicolumn{3}{c}{Profit} \\
\hline
Actif SJ & Acheteur & Vendeur \\
\hline
$S_T > K$ & $S_T - K - C(K,T)(1+r_f)^T$ & $K-S_T+C(K,T)(1+r_f)^T$\\
$S_T<K$ & $- C(K,T)(1+r_f)^T$ & $C(K,T)(1+r_f)^T$\\
\hline
\end{tabular}		


\includegraphics[scale=1]{_bookdown_files/03-GRF-2_files/figure-latex/unnamed-chunk-4-1.pdf}		


\begin{tabularx}{0.5\textwidth}{Xr}
\hline
\multicolumn{2}{c}{Valeur à l'échéance}\\
\hline
Acheteur & $max(0;S_T-K)$\\
Vendeur & $-max(0;S_T-K)$\\
\hline
\end{tabularx}

\subsection{Option de vente(put)}\label{option-de-venteput}

Contrat qui permet au détenteur(position courte) de vendre un actif
sous-jacent à un prix prédéterminé, strike price \(=K\), à une date
d'échéance ou d'içi cette date, s'il le désire. Le vendeur(position
longue) de l'option devra acheter le SJ à ce prix si le
détenteur(acheteur) le désire.\\
\\


\begin{tabular}{ccc}
\hline
\multicolumn{3}{c}{Option de vente} \\  
\hline
Position & Profit & Valeur à l'échéance \\
\hline
Acheteur & $max(0;K-S_T)-P(K,T)(1+r_f)^T$ & $max(0;K-S_T)$ \\
\\
Vendeur & $P(1+r_f)^T-max(0;K-S_T)$ & $-max(0;K-S_T)$\\
\hline
\end{tabular}		


\includegraphics[scale=1]{_bookdown_files/03-GRF-2_files/figure-latex/unnamed-chunk-5-1.pdf}


\section{Stratégies de couverture et spéculation sur volatilité}

\subsection{Floor}\label{floor}

Combinaison d'une postion longue dans le SJ(on le possède) et une
position courte dans une option de vente(achat). Permet de se couvrir
contre une baisse du prix du SJ.\\
\\
	


\begin{tabular}{|c|c|}
\hline 
\rule[-1ex]{0pt}{2.5ex} Valeur à l'échéance & $=S_T+\max(0;K-S_T)=\max(S_T;K)$ \\ 
\hline 
\rule[-1ex]{0pt}{2.5ex} Profit & $=\max(S_T,K)-(S_0+P(T,K))(1+r_f)^T$ \\ 
\hline 
\end{tabular}

\subsection{Vente de couverture:vendre un floor(option de vente couverte)}\label{vente-de-couverturevendre-un-flooroption-de-vente-couverte}

Combinaison d'une position longue dans l'option de vente(vente) et d'une
position courte dans le SJ(vente à découvert)\\
\\


\begin{tabular}{|c|c|}
\hline 
\rule[-1ex]{0pt}{2.5ex} Valeur à l'échéance & $=-S_T-\max(0;K-S_T)=-\max(S_T;K))$ \\ 
\hline 
\rule[-1ex]{0pt}{2.5ex} Profit & $=-\max(S_T,K)+(S_0+P(T,K))(1+r_f)^T$ \\ 
\hline 
\end{tabular}

\subsection{Cap}\label{cap}

Combinaison d'une position courte dans le SJ(vente à découvert)et d'une
position longue dans une option d'achat(achat).\\
\\

\begin{tabular}{|c|c|}
\hline 
\rule[-1ex]{0pt}{2.5ex} Valeur à l'échéance & $=S_T+\max(0;S_T-K)=\max(-S_T;-K)=-\min(S_T;K)$ \\ 
\hline 
\rule[-1ex]{0pt}{2.5ex} Profit & $=-\min(S_T,K)+(S_0-C(T,K))(1+r_f)^T$ \\ 
\hline 
\end{tabular}

\subsection{Vente de couverture:vendre un cap(option d'achat couverte)}\label{vente-de-couverturevendre-un-capoption-dachat-couverte}

Combinaison d'une position courte dans l'option d'achat(vente) et d'une
position longue dans le SJ(achat).\\
\\

\begin{tabular}{|c|c|}
\hline 
\rule[-1ex]{0pt}{2.5ex} Valeur à l'échéance & $=S_T-\max(0;S_T-K)=-\max(-S_T;-K)=\min(S_T;K)$ \\ 
\hline 
\rule[-1ex]{0pt}{2.5ex} Profit & $=\min(S_T,K)-(S_0-C(T,K))(1+r_f)^T$ \\ 
\hline 
\end{tabular}

\subsection{Foward synthétique}\label{foward-synthetique}

On fait un foward en combinant une position longue dans une option
d'achat et une position longue dans une option de vente avec la même
échéance et le même strike price.\\
\\

\begin{tabular}{|c|c|}
\hline 
Foward synthétique & $=Call(K,T)-Put(K,T)$ \\ 
\hline 
Coût initial & $C(K,T)-P(K,T)$ \\ 
\hline 
Valeur à l'échéance & $=\max(0;S_T-K)-\max(0;K-S_T)=S_T-K$ \\ 
\hline 
Profit & $(S_T-K)-(C(K,T)-P(K,T))(1+r_f)^T$ \\ 
\hline 
\end{tabular}\\
\\

Si on remplace \(K\) par \(F_{0,T}\), le prix d'exercice sera le même
qu'avec un foward standard. La différence avec un foward synthétique est
que \(K\not=F_{0,T}\) est possible et, ainsi, le coût initial ne sera
pas nul. Si \(K<F_{0,T}\), on payera le SJ moins cher, mais on payeune
prime initiale. Si \(K>F_{0,T}\), on payera plus cher le SJ, mais on
recevra une prime initiale.

\subsection{Parité des options d'achat et de vente}\label{parite-des-options-dachat-et-de-vente}

\begin{align*}
E\left [S_T-K-(C(K,T)-P(K,T)(1+r_f)^T\right ]& =E\left [\text{Profit}\right ]=0\\
E\left [S_T\right ]-K& =(C(K,T)-P(K,T))(1+r_f)^T\\
C(K,T)-P(K,T)& = (F_{0,T}-k)(1+r_f)^{-T}
\end{align*}

\subsection{Bull spread}\label{bull-spread}

\subsubsection*{Première façon:}\label{premiere-facon}

Combinaison d'une position longue dans une option d'achat à un prix
d'exercice \(K_1\) et d'une position courte dans une option d'achcat à
un prix d'exercice \(K_2,\;K_1<K_2\), avec la même date d'échéance.\\
\\

\begin{tabular}{|c|c|}
\hline 
Bull spread(call)& $=Call(K_2,T)-Call(K_1,T)$ \\ 
\hline 
Coût initial & $C(K_1,T)-C(K_2,T)$ \\ 
\hline 
Valeur à l'échéance & $=\max(0;S_T-K_1)-\max(0;S_T-K_2)$ \\ 
\hline 
Profit & $\max(0;S_T-K_1)-\max(0;S_T-K_2)-(C(K_1,T)-C(K_2,T))(1+r_f)^T$ \\ 
\hline 
\end{tabular}

\subsubsection*{Deuxième façon:}\label{deuxieme-facon}

Combinaison d'une position courte(achat) dans une option de vente à un
prix d'exercice \(K_1\) et d'une position longue(vente) dans une option
de vente à un prix d'exercice \(K_2,\;K_1<K_2\), avec la même date
d'échéance.\\
\\

\begin{tabular}{|c|c|}
\hline 
Bull spread(Put)& $=Put(K_2,T)-Put(K_1,T)$ \\ 
\hline 
Coût initial & $P(K_1,T)-P(K_2,T)$ \\ 
\hline 
Valeur à l'échéance & $=\max(0;K_1-S_T)-\max(0;K_2-S_T)$ \\ 
\hline 
Profit & $\max(0;K_1-S_T)-\max(0;K_2-S_T)-(P(K_1,T)-P(K_2,T))(1+r_f)^T$ \\ 
\hline 
\end{tabular}

\subsection{Bear Spread(-Bull spread)}\label{bear-spread-bull-spread}

\subsubsection*{Première façon:}\label{premiere-facon-1}

Combinaison d'une position courte(vente) dans une option d'achat à un
prix d'exercice \(K_1\) et d'une position longue(achat) dans une option
d'achcat à un prix d'exercice \(K_2,\;K_1<K_2\), avec la même date
d'échéance.\\
\\

\begin{tabular}{|c|c|}
\hline 
Bear spread(call)& $=Call(K_1,T)-Call(K_2,T)$ \\ 
\hline 
Coût initial & $C(K_2,T)-C(K_1,T)$ \\ 
\hline 
Valeur à l'échéance & $=\max(0;S_T-K_2)-\max(0;S_T-K_1)$ \\ 
\hline 
Profit & $\max(0;S_T-K_2)-\max(0;S_T-K_1)-(C(K_2,T)-C(K_1,T))(1+r_f)^T$ \\ 
\hline 
\end{tabular}

\subsubsection*{Deuxième façon:}\label{deuxieme-facon-1}

Combinaison d'une position longue(vente) dans une option de vente à un
prix d'exercice \(K_1\) et d'une position courte(achat) dans une option
de vente à un prix d'exercice \(K_2,\;K_1<K_2\), avec la même date
d'échéance.\\
\\

\begin{tabular}{|c|c|}
\hline 
Bull spread(Put)& $=Put(K_1,T)-Put(K_2,T)$ \\ 
\hline 
Coût initial & $P(K_2,T)-P(K_1,T)$ \\ 
\hline 
Valeur à l'échéance & $=\max(0;K_2-S_T)-\max(0;K_1-S_T)$ \\ 
\hline 
Profit & $\max(0;K_2-S_T)-\max(0;K_1-S_T)-(P(K_2,T)-P(K_1,T))(1+r_f)^T$ \\ 
\hline 
\end{tabular}

\subsection{Ratio spread}\label{ratio-spread}

Combinaison de n position longue(achat) dans les options d'achat à un
prix d'exercice \(K_1\) et m position courte(vente) dans les options
d'achat à un prix d'exercice \(K_2\), avec la m\^{}me date d'échéance.
Permet la possibilité de créer une combinaison quirésulte en un coût
initial nul.

\subsection{Box spread}\label{box-spread}

Combinaison de positions longue dans une option d'achat(achat) et de
vente(vente) à un prix d'exercice \(K_1\) et de positions courtes dans
une option d'achat(vente) et d'une option de vente(achat) à un prix
d'exercice \(K_2\), avec toutes les options de mêmes dates d'échéance.\\
\\

\begin{tabular}{|c|c|c|}
    \hline 
    Box spread &  $=Call(K_1,T)+Put(K_2,T)$ & $-Call(K_2,T)-Put(K_1,T)$ \\ 
    \hline 
    Box spread & Bull spread(call) & + Bear spread(put) \\ 
    \hline 
    Box spread & $=Call(K_1,T)-Call(K_2,T)$ & $+Put(K_2,T)-Put(K_1,T)$ \\ 
    \hline 
    Box spread & Foward synthétique $K_1$ & - Foward synthétique $K_2$  \\ 
    \hline 
    Box spread & $=Call(K_1,T)-Put(K_1,T)$ & $-(Call(K_2,T)-Put(K_2,T))$ \\ 
    \hline 
    Coût initial & $=C(K_1,T)+P(K_2,T)$ & $-C(K_2,T)-P(K_1,T)>0$ \\ 
    \hline 
    Valeur à l'échéance & $=\max(0;S_T-K_1)+\max(0;K_2-S_T)$ & $-\max(0;S_t-K_2)-\max(0;K_1-S_T)$ \\ 
    \hline 
    Profit & \multicolumn{2}{c|}{=0} \\
    \hline
\end{tabular}



On spécule sur la volatilité du SJ, de l'amplitude des variations de sa valeur, peu importe le sens de la variation.

\subsection{Straddles}

On achète une option d'achat et une option de vente au même prix d'exercice.
\[\text{Starddle}=\text{Call}(K)+\text{Put}(K)\]

\begin{enumerate}
\item Coût initial: $C(K,T) + P(K,T)>0$
\item Valeur à l'échéance: $\max(0,S_T-K)+ \max(0, K-S_T)= \vert S_T-K \vert$
\item Profit: $\vert S_T-K \vert - (C(K,T) + P(K,T))(1+r_f)^{T}$
\end{enumerate}

\subsection{Strangle}

On achète une option d'achat et une option de vente, mais avec un prix d'exercice différent. $K_1<K_2$ et en général $K_1<S_T<K_2$


\begin{enumerate}
\item Coût initial: $C(K_2,T) + P(K_1,T)>0$
\item Valeur à l'échéance: $\max(0,S_T-K_2)+ \max(0, K_1-S_T)$
\item Profit: $\max(0,S_T-K_2)+ \max(0, K_1-S_T) - (C(K_@,T) + P(K_1,T))(1+r_f)^{T}$
\end{enumerate}

\subsection{Butterfly spread}
Combinaison d'un straddle et d'un strangle. Le straddle est vendu au strike price $K_2$ et le strangle a des strikes prices de $K_1$ et $K_3$. Le but est de tirer profit d'une faible volatilité du SJ, mais en éliminant les risques liés aux pertes en cas de très forte volatilité.
\[\text{Butterfly spread}= \underbrace{(C(K_3,T)+P(K_1,T))}_{\text{Strangle}(K_1,K_3)}-\underbrace{(C(K_2,T)+P(K_2,T))}_{\text{Written Straddle}(K_2)}\]

\begin{enumerate}
\item Coût initial: $C(K_3,T) + P(K_1,T)-(C(K_2,T) + P(K_2,T))<0$
\item Valeur à l'échéance: $\max(0,S_T-K_3)+ \max(0, K_1-S_T)-(\max(0,S_T-K_2)+ \max(0, K_2-S_T))$
\item Profit: $\max(0,S_T-K_3)+ \max(0, K_1-S_T)- \vert S_T-K_2 \vert -C(K_3,T) + P(K_1,T)-(C(K_2,T) + P(K_2,T))(1+r_f)^{T}$
\end{enumerate}

\subsection{Butterfly spread asymétrique}

Combinaison de $n$ Bull spread($K_1,K_2$) et de $m$ Bear spread($K_2,K_3$). 
